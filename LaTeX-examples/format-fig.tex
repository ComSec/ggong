\documentclass{article}



\usepackage{amsmath,amssymb}
\usepackage{graphicx}
\usepackage{color}


\renewcommand{\baselinestretch}{1}
\renewcommand{\arraystretch}{1.3}



%you may inactive the following four commands, then it will become appearance of book pages

\setlength{\topmargin}{-0.1in}
\setlength{\textheight}{8.3in}
\setlength{\oddsidemargin}{0.1 in}
\setlength{\textwidth}{6.2 in}

%%%%%%%%%


%%%%%%%%%%%%%

\newtheorem{fact}{Fact}
\newtheorem{algorithm}{Algorithm}
\newtheorem{theorem}{Theorem}
\newtheorem{lemma}{Lemma}
\newtheorem{corollary}{Corollary}
\newtheorem{property}{Property}
\newtheorem{definition}{Definition}
\newtheorem{proposition}{Proposition}
\newtheorem{remark}{Remark}
\newtheorem{conjecture}{Conjecture}

\newcommand{\notsim}{{\, \not \sim \,}}

\newcommand{\F}{\ensuremath{\mathbb F}}
\newcommand{\Z}{\ensuremath{\mathbb Z}}
\newcommand{\N}{\ensuremath{\mathbb N}}
\newcommand{\Q}{\ensuremath{\mathbb Q}}
\newcommand{\R}{\ensuremath{\mathbb R}}
\newcommand{\C}{\ensuremath{\mathbb C}}

\newcommand{\done}{\hfill $\Box$ }
\newcommand{\rmap}{\stackrel{\rho}{\leftrightarrow}}
\newcommand{\mys}{\vspace{0.15in}}

\newcommand{\ebu}{{\bf{e}}}
\newcommand{\Abu}{{\bf{A}}}
\newcommand{\Bbu}{{\bf{B}}}

\newcommand{\abu}{{\bf{a}}}
\newcommand{\abuj}{{\bf{a_j}}}
\newcommand{\bbu}{{\bf{b}}}
\newcommand{\vbu}{{\bf{v}}}
\newcommand{\ubu}{{\bf{u}}}
\newcommand{\dbu}{{\bf{d}}}
\newcommand{\tbu}{{\bf{t}}}
\newcommand{\cbu}{{\bf{c}}}
\newcommand{\kbu}{{\bf{k}}}
\newcommand{\hbu}{{\bf{h}}}
\newcommand{\sbu}{{\bf{s}}}
\newcommand{\wbu}{{\bf{w}}}
\newcommand{\xbu}{{\bf{x}}}
\newcommand{\ybu}{{\bf{y}}}
\newcommand{\zbu}{{\bf{z}}}


\newcommand{\corres}{\leftrightarrow}

\newcommand{\zero}{{\bf{0}}}
\newcommand{\one}{{\bf{1}}}
\newcommand{\nodiv}{{\, \not| \,}}
\newcommand{\notequiv}{{\,\not\equiv\, }}


%%%%%%%%%%%%%%

\def\comb#1#2{{#1 \choose #2}}
\newcommand{\ls}[1]
    {\dimen0=\fontdimen6\the\font\lineskip=#1\dimen0
     \advance\lineskip.5\fontdimen5\the\font
     \advance\lineskip-\dimen0
     \lineskiplimit=0.9\lineskip
     \baselineskip=\lineskip
     \advance\baselineskip\dimen0
     \normallineskip\lineskip\normallineskiplimit\lineskiplimit
     \normalbaselineskip\baselineskip
     \ignorespaces}

\def\stir#1#2{\left\{#1 \atop #2 \right\}}
\def\dsum#1#2{#1 \atop #2 }
\def\defn{\stackrel{\triangle}{=}}

%%%%%%%%


\begin{document}

\bibliographystyle{abbrv}

\title{Figures in Latex Files }
\author{Guang Gong\\
Department of Electrical and Computer Engineering \\
University of Waterloo \\
Waterloo, Ontario N2L 3G1, CANADA \\
Email. ggong@calliope.uwaterloo.ca\\
}

\date{}
 \maketitle

\thispagestyle{plain}
\setcounter{page}{1}

\begin{abstract}
Observing a  phenomenon  ... 


{\bf Index Terms.}  Multiple signal sets, maximum correlation, bent functions, 
high-order shift-distinct property, CDMA.

\end{abstract}

\ls{1.5}
\section{How to insert a figure into a latex file}

Using xfig to draw figures,   see the following examples.


\begin{figure}[t]
\input{lfsr_gen.pstex_t}
\centering
\caption{LFSR implementation of ${\cal S}_o (\rho)$}
\label{fig:fig1}
\end{figure}

\begin{figure}[t]
\input{lfsr.pstex_t}
\centering
\caption{LFSR implementation of ${\cal S}_o (2)$ for $n =7$ with characteristic polynomials
$g_0(x) = x^7+x+1, g_1(x) = x^7+x^5+x^3+x+1$, $g_2(x) = x^7+x^3+x^2+x+1$, and $g_3(x) = x^7+x^5+x^4+x^3+1$.}
\label{fig:fig2}
\end{figure}


 
\end{document}


